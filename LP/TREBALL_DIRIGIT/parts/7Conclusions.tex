\newpage
\section{Conclusió}
Com s'ha pogut veure en aquest treball, l'automatització en la gestió de les regions de memòria dinàmica aporten coses bones i dolentes, depenent de l'ús de l'aplicació.

Si tenim una aplicació on els temps de resposta és important (per exemple aplicacions de temps real) és recomanable no utilitzar sistemes de \textit{garbage collection}, ja que podem perdre interaccions amb els usuaris. De forma anàloga per aplicacions on es busca un màxim rendiment. Si aquest fet no importa, podem utilitzar sistemes automàtics.

D'altra banda, per fer prototipat ràpid de codi/aplicacions; els recollidors de memòria brossa simplifiquen el codi i el seu desenvolupament. Un cop fet un prototipat, transferir els algoritmes d'un codi amb recolectors cap a un codi on el programador és responsable del \textit{heap} és mitjanament senzill.

Per tant, podem concloure que els algoritmes de \textit{garbage collection} treuen temps d'execucció als programes i s'ha de tenir cura en quin context s'utilitzen. Molts cops es poden modificar paràmetres i versions dels recollidors per ajustar-ho el mès possible a les nostres aplicacions.